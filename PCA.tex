\section{数据降维-主成分分析}
\subsection{PCA 算法}
主成分分析(Principal Component Analysis,PCA)是最常用的一种降维方法。

对于正交属性空间中的样本点,如果用一个超平面对所有样本进行恰当地表达,它应当具有下列性质:

\begin{itemize}
    \item 最近重构性:样本点到这个超平面的距离都足够近;
    \item 最大可分性:样本点到这个超平面上的投影能尽可能分开。
\end{itemize}

\begin{algorithm}[H]
    \renewcommand{\algorithmicrequire}{\textbf{Input:}}
	\renewcommand{\algorithmicensure}{\textbf{Output:}}
    \caption{PCA 算法}
    \begin{algorithmic}[1]
        \REQUIRE 样本集 $D = \left\{\boldsymbol{x}_1, \boldsymbol{x}_2, \cdots, \boldsymbol{x}_m\right\}$, 低维空间维数 $d'$
        \ENSURE 投影矩阵 $W = \left\{\boldsymbol{w}_1, \boldsymbol{w}_2, \cdots, \boldsymbol{w}_{d'}\right\}$
        \STATE 对所有样本进行中心化:$\boldsymbol{x}_i \leftarrow \boldsymbol{x}_i - \frac 1m \sum_{i=1}^m \boldsymbol{x}_i$
        \STATE 计算样本的协方差矩阵 $\mathbf{XX}^\mathrm T$
        \STATE 对协方差矩阵 $\mathbf{XX}^\mathrm T$ 做特征值分解
        \STATE 选取前 $d'$ 个最大的特征值对应的特征向量作为投影矩阵 $W$
    \end{algorithmic}
\end{algorithm}

\subsection{特征值分解}
特征值分解:

\begin{itemize}
    \item 先求 $A$ 的特征多项式 $f(\lambda) = |A - \lambda E| = |A - \Lambda|$,
    其中 $\Lambda$ 是一个主对角线上全为 $\lambda$ 的方阵;
    \item 求特征方程 $|A - \lambda E| = 0$ 的全部解,他们就是 $A$ 的全部特征值;
    \item  对于一个特征值 $\lambda_i$,求出相应的特征方程组 $(A - \Lambda_i)X = 0$ 的一组特解
     $\xi_1, \xi_2, \cdots, \xi_t$,则这组解对应的向量即为一个特征向量。
\end{itemize}

降维后低维空间的维数 $d'$ 通常是由用户事先指定,
或通过在 $d'$ 值不同的低维空间中对 k 近邻分类器(或其他开销较小的学习器)进行交叉验证来选取较好的 $d'$ 值。
对 PCA,还可以还可从重构的角度设置一个重构阈值,例如 $t = 95\%$,然后选取使下式成立的最小 $d'$ 值:
\begin{equation}
   \dfrac{\sum_{i=1}^{d'} \lambda_i}{\sum_{i=1}^d{\lambda_i}} \geq t 
\end{equation}


PCA 仅需保留 $\mathbf W$ 与样本的均值向量,即可通过简单的向量减法和矩阵-向量乘法将新样本投影至低维空间中。

降维虽然会导致信息的损失,但一方面舍弃这些信息后能\textbf{使得样本的采样密度增大},
另一方面,当数据受到噪声影响时,最小的特征值所对应的特征向量往往与噪声有关,\textbf{舍弃可以起到去噪效果}。