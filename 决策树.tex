\section{决策树}
\subsection{基本算法流程}
\begin{algorithm}[H]
    \renewcommand{\algorithmicrequire}{\textbf{Input:}}
	\renewcommand{\algorithmicensure}{\textbf{Output:}}
    \caption{决策树学习基本算法 $\text{TreeGenerate(D, A)}$}
    \begin{algorithmic}[1]
        \REQUIRE 训练集 $D = \left\{(x_1, y_1), (x_2, y_2), \dots, (x_m, y_m)\right\}$ \\ \qquad 属性集 $A = \left\{a_1, a_2, \dots, a_d\right\}$
        \ENSURE 以 node 为根结点的一棵决策树
        \STATE 生成结点 node;
        \IF{$D$ 中样本全属于同一类别 $C$}
        \STATE 将 node 标记为 $C$ 类叶结点;
        \STATE \textbf{return}
        \ENDIF
        \IF{$A = \varnothing$ \textbf{OR} $D$ 中样本在 $A$ 上取值相同}
        \STATE 将 node 标记为叶结点,其类别标记为 $D$ 中样本数最多的类;
        \STATE \textbf{return}
        \ENDIF
        \STATE 从 $A$ 中选择最优划分属性 $a_*$;
        \FOR{$a_*$ 中的每一个值 $a_*^v$}
        \STATE 为 node 生成一个分支;
        \STATE 令 $D_v$ 表示 $D$ 中在 $a_*$ 上取值为 $a_*^v$ 的样本子集;
        \IF{$D_v$ 为空}
        \STATE 将分支结点标记为叶结点,其类别标记为 $D$ 中样本最多的类;
        \STATE \textbf{return}
        \ELSE
        \STATE 以 $\text{TreeGenerate}\left(D_v, A \backslash \left\{a_*\right\}\right)$ 为分支结点;
        \ENDIF
        \ENDFOR
    \end{algorithmic}
\end{algorithm}
\subsection{划分选择}
\subsubsection{信息增益}
令信息熵
\begin{equation}
    \operatorname{Ent}(D) = -\sum\limits_{k=1}^{|\mathcal{Y}|}p_k\log _2(p_k)
\end{equation}
假设离散属性 $a$ 有 $V$ 个可能的取值 $\left\{a^1, a^2, \dots, a^V\right\}$,如果使用属性 $a$ 为样本集进行划分,则会产生 $V$ 个分支结点,
其中第 $v$ 个分支结点包含了 $D$ 中所有属性为 $a^v$ 的样本,记为 $D^v$。我们用下式表示属性 $a$ 对样本集 $D$ 进行划分所获得的“信息增益”:
\begin{equation}
    \operatorname{Gain}(D, a) = \operatorname{Ent}(D) - \sum\limits_{v = 1}^V
    {\dfrac{|D^v|}{|D|}\operatorname{Ent}(D^v)}
\end{equation}

选择最优属性时我们可以采用属性 $a_* = \underset{a \in A}{\arg\max}\operatorname{Gain}(D, a)$。

\subsubsection{增益率}
由于信息增益准则对\textbf{可取值数目较多}的属性有所偏好,为了避免这种影响,
由 Quinlan 在 1993 年提出的 C4.5 决策树算法不直接使用信息增益,而是使用“增益率”来选择最优划分属性,定义为:
\begin{equation}
    \operatorname{Gain\_ratio}(D, a) = \dfrac{\operatorname{Gain}(D, a)}{\operatorname{IV}(a)}
\end{equation}
其中,$\operatorname{IV}(a)$ 表示属性 $a$ 的“固有值”。
属性 $a$ 的可能取值数目越多(即 $V$ 越大),则 $\operatorname{IV}(a)$ 的值通常会越大:
\begin{equation}
    \operatorname{IV}(a) = -\sum\limits_{v = 1}^V
    {\dfrac{|D^v|}{|D|}\log _2\dfrac{|D^v|}{|D|}}
\end{equation}

需注意的是,增益率准则对\textbf{可取值数目较少}的属性有所偏好,因此,
C4.5 算法采用了一种启发式的做法:先找出信息增益高于平均水平的属性,再从其中选择增益率最高的。

\subsubsection{基尼指数}
数据集 $D$ 的纯度可用基尼值来度量:
\begin{equation}
    \begin{aligned}
        \operatorname{Gini}(D) &= \sum\limits_{k=1}^{|\mathcal Y|}{\sum\limits_{k' \neq k}p_kp_{k'}} \\
        &=1 - \sum\limits_{k=1}^{|\mathcal Y|} p_k^2
        \end{aligned}
\end{equation}
直观上,$\operatorname{Gini}(D)$ 反映了从数据集 $D$ 中随机抽取两个样本,
其类别标记不一致的概率,所以 $\operatorname{Gini}(D)$ 越小,则数据集 $D$ 的纯度越高。

属性 $a$ 的基尼指数定义为
\begin{equation}
    \operatorname{Gini\_index}(D, a) = \sum\limits_{v = 1}^V
    {\dfrac{|D^v|}{|D|}\operatorname{Gini}(D^v)}
\end{equation}

我们选择划分后基尼指数最小的属性作为最优划分属性,即 $a_* = \underset{a \in A}{\arg\min}\operatorname{Gini\_index}(D, a)$。

\subsection{剪枝}
剪枝分为“预剪枝”和“后剪枝”两种。

“预剪枝”是指在决策树生成过程中,对每个结点在划分前进行估计,如果划分后泛化能力无法提升,则停止划分并将当前结点标记为叶结点;

“后剪枝”则是先从训练集生成一棵完整的决策树,然后自下向上地对非叶结点进行考察,
如果把该结点的子树替换为叶结点能提高泛化能力,则将该子树替换为叶结点。

对比:
\begin{itemize}
    \item 时间开销:
    \begin{itemize}
        \item 预剪枝:训练时间开销降低,测试时间开销降低
        \item 后剪枝:训练时间开销增加,测试时间开销降低
    \end{itemize}
    \item 过/欠拟合风险:
    \begin{itemize}
        \item 预剪枝:过拟合风险降低,欠拟合风险增加
        \item 后剪枝:过拟合风险降低,欠拟合风险基本不变
    \end{itemize}
    \item 泛化性能:后剪枝通常优于预剪枝
\end{itemize}

\subsection{连续值处理}
设给定样本集 $D$ 和连续属性 $a$,$a$ 在 $D$ 上出现了 $n$ 个不同的取值,
从小到大排序后记为 $\left\{a^1, a^2, \dots, a^n\right\}$。

我们可考察包含 $n - 1$ 个元素的候选划分点集合(即将区间的中点作为候选划分点)
\begin{equation}
    T_a = \left\{\dfrac{a^i + a^{i + 1}}{2}, 1 \leq i \leq n - 1\right\}
\end{equation}

对于信息增益,我们可以稍加改造得到属性连续时的公式:
\begin{equation}
    \begin{aligned}
        \operatorname{Gain}(D, a) &= \underset{t \in T_a}{\max} \operatorname{Gain}(D, a, t) \\
        &= \underset{t \in T_a}{\max} \operatorname{Ent}(D) - \sum\limits_{\lambda \in \left\{-, +\right\}}{\dfrac{|D_t^\lambda|}{|D|}\operatorname{Ent}(D_t^\lambda)}
        \end{aligned}
\end{equation}

\subsection{缺失值处理}
缺失值的处理涉及到两个问题:

(1)如何在属性值缺失的情况下进行划分属性选择? 

(2)给定划分属性,若样本在该属性上的值缺失,如何对样本进行划分?

给定训练集 $D$ 和属性 $a$,令 $\tilde{D}$ 表示 $D$ 中在属性 $a$ 上没有缺失值的样本子集。对于问题(1),我们可以仅通过 $\tilde{D}$ 来判断属性的优劣。

假定 $a$ 有 $V$ 个可取值 $\left\{a^1, a^2, \dots, a^V\right\}$,令 $\tilde{D}^v$ 表示 $\tilde{D}$ 
中在属性 $a$ 上取值为 $a^v$ 的样本子集,$\tilde{D}_k$ 表示 $\tilde{D}$ 
中属于第 $k$ 类($k = 1, 2, \dots, |\mathcal Y|$)的样本子集,则显然有
\begin{equation}
    \begin{aligned}
        \tilde{D} &= \bigcup\limits_{k=1}^{|\mathcal Y|} \tilde{D}_k \\
        \tilde{D} &= \bigcup\limits_{v=1}^V \tilde{D}^v
        \end{aligned}
\end{equation}

假定我们为每个样本 $x$ 赋予一个权重 $w_x$,并定义
\begin{equation}
    \begin{aligned}
        \rho &= \dfrac{\sum_{x \in \tilde{D}}w_x}{\sum_{x \in D} w_x} \\
        \tilde{p}_k &= \dfrac{\sum_{x \in \tilde{D}_k} w_x}{\sum_{x \in D} w_x}\quad (1 \leq k \leq |\mathcal Y|) \\
        \tilde{r}_v &= \dfrac{\sum_{x \in \tilde{D}^v} w_x}{\sum_{x \in D} w_x}\quad (1 \leq v \leq V)
        \end{aligned}
\end{equation}

$\rho$ 表示无缺失值样本所占的比例,$\tilde{p}_k$ 表示无缺失值样本中第 $k$ 类所占比例,
$\tilde{r}_v$ 表示无缺失值样本中在属性 $a$ 上取值为 $a^v$ 的样本所占的比例。

基于上述定义,我们可将信息增益的计算式推广为
\begin{equation}
  \begin{aligned}
\operatorname{Gain}(D, a) &= \rho \times \operatorname{Gain}(\tilde{D}, a) \\
&= \rho \times \left(\operatorname{Ent}\left(\tilde{D}\right) - \sum\limits_{v=1}^V{\tilde{r}_v\operatorname{Ent}\left(\tilde{D}^v\right)}\right)
\end{aligned}  
\end{equation}

其中
\begin{equation}
    \operatorname{Ent}(\tilde{D}) = -\sum\limits_{k=1}^{|\mathcal Y|}{\tilde{p}_k \log_2 \tilde{p}_k}
\end{equation}

对问题(2):

\begin{itemize}
    \item 若样本 $x$ 在划分属性 $a$ 上的取值已知,则将 $x$ 划入与其取值对应的子结点,且样本权值在子结点中保持为 $w_x$;
    \item 若样本 $x$ 在划分属性 $a$ 上的取值未知,则将 $x$ 同时划入所有子结点,
    且样本权值在与属性值 $a^v$ 对应的子结点中调整为 $\tilde{r}_v \cdot w_x$。
\end{itemize}

\subsection{从“树”到“规则”,从“单变量”到“多变量”}

可以发现,一颗决策树对应着一个“规则集”,每个从根结点到叶结点的分支路径对应于一条规则。
将决策树转化为一个规则集可以改善可理解性,进一步提升泛化能力。

在每个非叶结点仅考虑一个划分属性时得到的决策树称为“\textbf{单变量决策树}”,产生的是“轴平行”分类面,
当分类边界比较复杂时,单变量决策树必须使用很多段划分才能获得较好的近似,此时决策树会相当复杂,时间开销会很大。
若能够使用斜的划分边界,则模型将大为简化,“多变量决策树”可以做到这一点。
在该模型中,每个非叶结点不再是仅对某个属性,而是\textbf{构造一个线性分类器},如此决策树模型可以大为简化。
更复杂地,我们可以在结点上嵌入神经网络或其他非线性模型,获得更优的决策边界。