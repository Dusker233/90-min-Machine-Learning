\section{Softmax 回归}

Softmax 回归用于多分类问题,如果分类的类别是互斥的,适于选择 softmax 回归分类器。
如果类别之间不是互斥的,而是相互掺杂,则 k 个 logistic 回归分类器更加合适。当 $k = 2$ 时,Softmax 回归退化为逻辑回归。

\subsection{假设函数}
\begin{equation}
    h_\theta\left(x^{(i)}\right) = 
\begin{bmatrix}
p(y^{(i)}=1 \mid x^{(i)};\theta) \\
p(y^{(i)}=2 \mid x^{(i)};\theta) \\
\vdots \\
p(y^{(i)}=k \mid x^{(i)};\theta)
\end{bmatrix}
= \dfrac{1}{\sum_{j=1}^k e^{\theta_j^\mathrm Tx^{(i)}}}
\begin{bmatrix}
e^{\theta_1^\mathrm Tx^{(i)}} \\
e^{\theta_2^\mathrm Tx^{(i)}} \\
\vdots \\
e^{\theta_k^\mathrm Tx^{(i)}}
\end{bmatrix}
\end{equation}

其中 $\theta_1, \theta_2, \cdots, \theta_k \in \mathbb{R}^{n+1}$。

\subsection{代价函数}
若令 $\boldsymbol{\theta} =
\begin{bmatrix}
\theta_1^\mathrm T \\
\theta_2^\mathrm T \\
\vdots \\
\theta_k^\mathrm T
\end{bmatrix}_{k \times (n + 1)}$,则代价函数为
\begin{equation}
    J(\boldsymbol{\theta}) = -\dfrac 1m \left(\sum_{i=1}^m\sum_{j=1}^k1\left[y^{(i)} = j\right]\log 
    \dfrac{e^{\theta_j^\mathrm Tx^{(i)}}}{\sum_{l=1}^k e^{\theta_l^\mathrm Tx^{(i)}}}\right)
\end{equation}

其中 $1[x]$ 为示性函数,表示 $x$ 为真时为 $1$,否则为 $0$。

\subsection{梯度下降}
在 Softmax 回归中,将 $x$ 归类为类别 $j$ 的概率为
\begin{equation}
    p\left(y^{(i)}=j \mid x^{(i)} ; \theta\right)
    =\frac{e^{\theta_{j}^{\mathrm T} x^{(i)}}}{\sum_{l=1}^{k} e^{\theta_{l}^{\mathrm T} x^{(i)}}}
\end{equation}

则梯度公式为
\begin{equation}
    \nabla_{\theta_{j}} J(\boldsymbol\theta)=-\frac{1}{m} 
    \sum_{i=1}^{m}\left[x^{(i)}\left(1\left[y^{(i)}=j\right]-p\left(y^{(i)}=j \mid x^{(i)} ; \theta\right)\right)\right]
\end{equation}

\subsection{权重衰减}
所谓权重衰减,事实上就是正则化。
\begin{equation}
    J(\boldsymbol{\theta}) = -\dfrac 1m \left(\sum_{i=1}^m\sum_{j=1}^k1\left[y^{(i)} = j\right]\log \dfrac{e^{\theta_j^\mathrm Tx^{(i)}}}{\sum_{l=1}^k 
    e^{\theta_l^\mathrm Tx^{(i)}}}\right) + \textcolor{red}{\dfrac \lambda 2\sum_{i=1}^k\sum_{j=0}^n\theta_{ij}^2}
\end{equation}

此时梯度公式为
\begin{equation}
    \nabla_{\theta_{j}} J(\boldsymbol\theta)=-\frac{1}{m} \sum_{j=1}^{m}
    \left[x^{(i)}\left(1\left[y^{(i)}=j\right]-p\left(y^{(i)}=j \mid x^{(i)} ; \theta\right)\right)\right] 
    + \textcolor{red}{\lambda \theta_j}
\end{equation}

\begin{example}
    证明 logistic 回归是 softmax 回归在 $k = 2$ 时的特例。
\end{example}

\begin{proof}
    令 $k = 2$,则
    \begin{equation}
        h_\theta\left(x^{(i)}\right) = 
        \dfrac{1}{e^{\theta_1^\mathrm Tx^{(i)}} + e^{\theta_2^\mathrm Tx^{(i)}}}
        \begin{bmatrix}
        e^{\theta_1^\mathrm Tx^{(i)}} \\
        e^{\theta_2^\mathrm Tx^{(i)}}
        \end{bmatrix}
    \end{equation}

    令 $\theta_1$,$\theta_2$ 均减去 $\theta_1$,得到
    \begin{equation}
        \begin{aligned}
            h_\theta\left(x^{(i)}\right) &= 
        \dfrac{1}{e^{\vec 0^\mathrm Tx^{(i)}} + e^{(\theta_2-\theta_1)^\mathrm Tx^{(i)}}}
        \begin{bmatrix}
        e^{\vec 0^\mathrm Tx^{(i)}} \\
        e^{(\theta_2 - \theta_1)^\mathrm Tx^{(i)}}
        \end{bmatrix} \\
        &= \dfrac{1}{1 + e^{(\theta_2 - \theta_1)^\mathrm Tx^{(i)}}}
        \begin{bmatrix}
        1 \\
        e^{(\theta_2 - \theta_1)^\mathrm Tx^{(i)}}
        \end{bmatrix} \\
        &= \begin{bmatrix}
            \dfrac 1{1 + e^{(\theta_2 - \theta_1)^\mathrm Tx^{(i)}}} \\
            \dfrac{e^{(\theta_2 - \theta_1)^\mathrm Tx^{(i)}}}{1 + e^{(\theta_2 - \theta_1)^\mathrm Tx^{(i)}}}
        \end{bmatrix} \\
        &= \begin{bmatrix}
            1 - p \\
            p
        \end{bmatrix}.
        \end{aligned}
    \end{equation}
\end{proof}