\section{逻辑回归}

\subsection{Sigmoid 函数}
Sigmoid 函数为
\begin{equation}
    g(x) = \dfrac{1}{1 + e^{-x}}
\end{equation}

性质:$g(x) \in (0, 1)$,$g(0) = \dfrac 12$,$\lim\limits_{x \to -\infty}g(x) = 0$,$\lim\limits_{x \to +\infty}g(x) = 1$。

\subsection{预测函数}
\begin{equation}
    h_\theta(\boldsymbol{x}) = g(\boldsymbol{\theta}^\mathrm T\boldsymbol{x} + b)
\end{equation}
    
\subsection{代价函数}
\begin{equation}
    J(\boldsymbol{\theta}) = -\dfrac{1}{m}\sum_{i=1}^m\left[y^{(i)}\log h_\theta(\boldsymbol{x}^{(i)}) + (1 - y^{(i)})\log(1 - h_\theta(\boldsymbol{x}^{(i)}))\right]
\end{equation}

此代价函数称为\textbf{对数似然损失函数},也称为\textbf{交叉熵损失函数}。

此时梯度公式为
\begin{equation}
    \nabla J(\theta_j) = \dfrac{1}{m}\sum_{i=1}^m(h_\theta(\boldsymbol{x}^{(i)}) - y^{(i)})\boldsymbol{x}^{(i)}_j
\end{equation}

\subsection{多分类问题}
对于多分类问题,可以使用一对多的方法,即对于每个类别 $k$,训练一个逻辑回归分类器。
对于一个输入,只需找到输出最大的那个分类器,并将其对应的类别作为输入的类别。

对于第 $i$ 个分类器,输出的意义是\textbf{输入 $x$ 属于第 $i$ 类的概率为 $h^i_\theta(x)$}。